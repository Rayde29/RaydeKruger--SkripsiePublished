\chapter{Responsible Resources Use and End-of-life Strategy}
\label{sec: Resources}
    \section{Responsible Resources Use}
        During the design process, the parts were optimized to minimize material waste. The 3D prints were optimized to reduce the amount of support material required, this was done by avoiding overhangs and adding chamfers into the design to make it more 3D printer friendly. The components sent to the workshop were designed to minimize the waste material that would be machined away from the part. The proof of concept is primarily 3D printed, which is an energy efficient method for parts manufacturing.\\
        Responsible use of finical resources was applied. All components purchased from outside the university remained inside the allocated budget. In cases where the sensors were too expensive, creative solutions were used to have the same effect at a lower price.

    \section{End-of-life strategy}
        Engineers need to ensure each of their projects have an end of life strategy to help ensure sustainable development. For this project a proof of concept will be developed. Most of this proof of concept will be 3D printed in ABS, which is 100\% recyclable. The proof of concept can easily be disassembled with a screwdriver. The circuit boards were all connected with JST connections which can be easily removed, and the motors are connected to the aircraft with 3 pin connectors, which can be easily disconnected. This makes it easy separation of the structure and the electronics. The more expensive electronic components, such as the transceivers and microcontrollers have been connected using headers and so can easily be removed for use in future projects. The motors and load cells can also be disconnected and disassembled from the aircraft to be used elsewhere, alternately the aircraft can be removed from the test bench and the test bench can be used for other experiments. These electrical components, such as the transceiver, motor, and load cells can all be easily implemented in other projects. The metal rods in the rotors are mostly not machined and once removed can be used to create components smaller than 590~mm. The design of the aircraft was done with disassembly and recycling in mind to ensure that the proof of concept contributes to a safe and suitable future.

% \begin{table}[h]
%     \centering
%     \begin{tabular}{l  c r}
%         \toprule
%         Column 1 & Column 2 & Column 3 \\
%         \midrule
%         Data 1   & 123      & 456      \\
%         Data 2   & 789      & 101      \\
%         Data 3   & 112      & 131      \\
%         \bottomrule
%     \end{tabular}
%     \caption{A better-looking table using the booktabs package}
%     \label{tab:my_table}
% \end{table}

        
%         Before the design will be made, a list of potential items that could be used for the project will be created. This list could include actuators,  power sources, microcontrollers, sensors and building materials. This list will be used during the design phase and a prototype that uses as many items from previous projects will be created, provided it does not affect its performance. While in the design phase, each piece will be created with the intentions of being able to disassembly the prototype. For all items that must be bought and cannot be reused, an active effort will be made to use recyclable materials and in the event of using a  recyclable  material, a disposal plan will be created, stating the precautions and necessary steps that were training to ensure a safe and sustainable development.
\pagebreak
    

        