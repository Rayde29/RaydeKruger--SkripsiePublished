\chapter{Project Definition}
\section{Problem statement}
    As previously mentioned this project will go through the research, design, building and testing of a tip-thrust rotary-aircraft. This project aims to create a tip-thrust rotary-aircraft for which the rotor's pitch is controlled by the tip thrust. An investigation will be made to identify how traditional methods, such as a swashplate are used and how this can be achieved using the tip thrusts. To achieve directional thrust the tip thrust should vary the pitch of the rotor such the rotor will have a higher pitch on one side, thereby increasing the lift generated, and a low pitch on the other side, this will induce directional thrust.

\section{Scope and limitations}
    The final design should prove controllability, but does not need to achieve sustained flight, and thus showing that the aircraft can produce lift in the desired direction will suffice. While a basic understanding of rotor design can be applied to the aircraft's main rotor, it is not the focus of the project and thus no computational fluid dynamics are required either. The most common method of propulsion for tip-thrust rotary aircraft is using an operating fluid in either a hot or cold cycle, this method will not be investigated due to the required large scale of these methods. 
\section{Objectives}
    The objectives of this project are as follows:
    \begin{enumerate}
        \item Create a mathematical model which describes the thrust generated
        \item Design a proof of concept to achieve the desired aim
        \item Construct a working proof of concept of the created design 
        \item Implement a method to produce directional thrust
        \item Implement a control system for the tip thrust
        \item Test and analyze the proof of concept
    \end{enumerate}
\section{Research questions}
    \subsection{Can the aircraft be fully controlled using the tip thrust alone?}
        An investigation should be done to test the viability of using tip thrust to introduce the control of the direction of the rotary aircraft.
    \subsection{How accurate is the mathematical model?}
        The amount of thrust produced should be compared to the amount of thrust predicted by the mathematical model to determine the accuracy of the model.
    \subsection{What is the efficiency of the aircraft?}
        The efficiency of the rotary aircraft can be evaluated by comparing the total lift it can produce compared to the lift created by the tip-thrust motors to determine if it is more effective than conventional drones.
    % \subsection{Which control system method works best?}
    %     An investigation of the different control systems, including PID, lead/lag compensator and state space control, should be looked into to determine which method is the most effective. 
    % \subsection{How stable is the system?}
    %     Rotary aircraft are inherently unstable. The system needs to be checked to determine how stable the implemented control system is.

\section{Motivation}
    As previously mentioned, tip-thrust rotary aircraft remove the need for a tail rotor as they do not produce a torque that needs to be canceled. This decreases the complexity of the aircraft and reduces its weight as there is no longer a need for large transmission shafts and gearboxes. However, many current designs use this method of propulsion for autogyro aircraft designs. These use the main rotor to produce lift and have other methods for directional thrust. This adds another system to the aircraft which could introduce unreliability and increase complexity. By making the aircraft's pitch controllable with the tip thrusts on the rotor, it will decrease the complexity of the aircraft and decrease the weight, allowing for a larger payload to be carried. Decreasing the weight can also increase the flight time of the aircraft. With the decreased weight and a larger rotor than conventional quad-copters, the efficiency of the aircraft will be increased, this will increase the potential flight time of the aircraft and reduce the mechanical complexity that is present with traditional helicopters, allowing the creation of small-scale aircraft with longer flight time a possibility. 
    
\section{Use of AI}
    During this project AI such as ChatGPT scholar was used to assist with gathering information and finding resources. Grammerly and ChatGPT were used to help improve grammar and the sentence structure, but no generated text has been used in the report. The STM32 Solver AI tool as well as ChatGPT was used to assist with debugging code as well as formatting for  MATLAB. A GUI (graphical user interface) was generated using ChatGPT. This interface was merely used to easily send commands and receive and display data easily. The proof of concept can function without this by using Termite to send and receive commands. The use of the GUI helped increase the efficiency for testing.  
    
    % Traditional shaft-drive helicopters need an engine-to-rotor power transmission to the main rotor and  tail rotor. The output shaft goes into a high rpm/ low torque gearbox and converts the input to low rpm/ high torque output to drive the rotor. The rate of cost and weight at which the transmission must increase is much higher than the rate helicopter that the helicopter increases in size and soon becomes a limiting factor. 
