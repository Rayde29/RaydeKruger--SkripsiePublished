\chapter{Project Scheduled Plan and ECSA Requirements }
    \section{End-of-life strategy}
        Engineers need to ensure each of their projects have an end of life strategy to help ensure sustainable development. For this project a prototype will be developed. Before the design will be made, a list of potential items that could be used for the project will be created. This list could include actuators,  power sources, microcontrollers, sensors and building materials. This list will be used during the design phase and a prototype that uses as many items from previous projects will be created, provided it does not affect its performance. While in the design phase, each piece will be created with the intentions of being able to disassembly the prototype. For all items that must be bought and cannot be reused, an active effort will be made to use recyclable materials and in the event of using a  recyclable  material, a disposal plan will be created, stating the precautions and necessary steps that were training to ensure a safe and sustainable development.
\pagebreak

        \section{ECSA Requirements}
        \captionof{table}{ECSA Requirements} % Use \captionof from caption package for captions outside float environments
        \label{tab:ecsa_requirements}
        \hspace{-2cm}
        \begin{tabular}{|c|>{\raggedright}p{3.5cm}|>{\raggedright}p{4cm}|p{8cm}|} \hline
            Nr   &    \textbf{ECSA Graduate Attribute (GA)}   &   \textbf{Activity addressing attribute}  &   \textbf{Reasoning} \\ \hline
            GA1 & \textbf{Problem-solving}  & 4.1.2,4.1.3,4.1.5 &  The planned activities mentioned all have clear objectives and obstacles that need to be overcome to achieve them. These will require creative solutions to achieve the objective. \\ \hline 
            GA2 & \textbf{Application of scientific and engineering knowledge} & 4.1.4, 4.1.5, 4.1.8 & These actives include creating a mathematical model of the system to then create a control system to be implemented. The building and design process will incorporate workshop trainings and learned from practicals  \\ \hline
            GA3 & E\textbf{ngineering design} & 4.1.1, 4.1.7 &   Each of these activities requires the student to create a solution to the problem by designing a part which will integrate into the system as a whole. \\ \hline
            GA4 & \textbf{Engineering methods, skills and tools, including Information Technology} & 4.1.2, 4.1.5, 4.1.6 & These activities all require skills gained throughout the Engineering course to be able to understand the system and to implement the appropriate solution. \\ \hline
            GA5 & \textbf{Professional and technical communication} & 4.1.11 & Professional report writing to explain the results and how the objectives were met will ensure all the student's information has been conveyed correctly. \\ \hline
            GA6 & \textbf{Individual, team and multi-disciplinary working} & 4.1.8, 4.1.10 & Throughout the build process, the student will need to work with technicians to help build the system. Effectively working with the technicians will ensure that the component will be created to the student's specification. The whole project will have the student's demonstrating their individual work. \\ \hline
            GA7 & \textbf{Independent learning ability} & 4.1.1, 4.1.2 &  As the student is a Mechatronic engineer, the more advanced fluid mechanics, such as rotor design, will require self work to research and understand how the rotor can be designed effectively. \\ \hline
        \end{tabular}
       

        